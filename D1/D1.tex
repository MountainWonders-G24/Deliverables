\documentclass[a4paper,12pt]{article}

%\begin{figure}[htp]
%    \centering
%    \includegraphics[width=0.75\textwidth]{}
%\end{figure}

%%%%%%%%%%%%%%%%%%%%
%%%%  PREAMBLE  %%%%
%%%%%%%%%%%%%%%%%%%%
\usepackage{float}
\usepackage[T1]{fontenc}
\usepackage[utf8]{inputenc}

\usepackage[english,italian]{babel}
\usepackage{graphicx}     % Per includere immagini
\usepackage{subcaption}   % Per utilizzare subfigure
\usepackage{hyperref}
\usepackage{indentfirst}

\hypersetup{hidelinks}

\usepackage[margin=2.5cm]{geometry}
\usepackage{minipage-marginpar}
\usepackage{fancyhdr}
\usepackage[bottom]{footmisc}
\usepackage{lastpage}

\usepackage{enumitem}
\usepackage{tabularx}

\usepackage{graphicx}

\setlength{\parindent}{0em}
\setlength{\parskip}{1em}

\fancyhead[L]{\leftmark}
\fancyhead[R]{\shortstack[r]{Versione documento: 0.01 \\ Gruppo: G24}}

\fancyfoot[C]{}
\fancyfoot[R]{\thepage/\pageref{LastPage}}

\renewcommand{\headrulewidth}{2pt}
\renewcommand{\headruleskip}{3pt}
\setlength{\headheight}{30pt}

\renewcommand{\footrulewidth}{2pt}

\setlist[itemize]{itemsep=0.25em,topsep=0pt}
\setlist[enumerate]{itemsep=0.25em,topsep=0pt,align=left}

%%%%%%%%%%%%%%%%%%%%
%%%%  DOCUMENT  %%%%
%%%%%%%%%%%%%%%%%%%%

\title{}
\author{Gruppo G24}

\begin{document}

\pagestyle{empty}

\begin{center}

    \vspace{2 cm}

    \begin{tabular*}{\textwidth}{ c @{\extracolsep{\fill}} c }
        \includegraphics[width=0.3\textwidth]{marchio_unitrento.pdf} & \shortstack{\Large{Dipartimento di Ingegneria} \\ \Large{e Scienza dell'Informazione}}
    \end{tabular*}

    \vspace{5 cm} 
  
    \Huge \textbf{Ingegneria del software\\}
  
    \vspace{1.5 cm} 
    \Large\textsc{Documento dei requisiti\\} 
    \vspace{3 cm} 
    \Huge\textsc{Mountain Wonders\\}
    \Large{Gruppo G24}
  
    \vspace{2 cm} 
  
    \Large{Anno accademico 2023/2024}
\end{center}

\newpage
\tableofcontents

\pagestyle{fancy}
\newpage
\section{Scopo del documento}

Il presente documento riporta l’analisi dei requisiti di sistema del progetto MountainWonders. Lo scopo di questo documento è quello di:
\begin{itemize}+
    \item presentare gli obiettivi del progetto;
    \item descrivere i requisiti funzionali;
    \item elencare i requisiti non funzionali;
    \item presentare il front-end del progetto;
    \item descrivere il back-end del progetto.
\end{itemize}

\section{Obiettivi del progetto}

Questo progetto ha l’obiettivo di realizzare una piattaforma web per la visualizzazione di rifugi e punti di spicco del Trentino - Alto Adige. In particolare si concentrerà sulla loro catalogazione e la possibilità da parte degli utenti di valutare determinati parametri di essi. 
Nello specifico l'utente che visita il sito sarà in grado di:

\begin{enumerate}
    \item visualizzare le montagne presenti in Trentino e i rifugi presenti su di esse
    \item recensire un rifugio in modo da poter invogliare altri utenti a visitarlo
    \item fare una ricerca tra i rifugi presenti
    \item gestione dei contenuti delle montagne
    \item aggiungere rifugi

\end{enumerate}
    
\newpage

\section{Requisiti funzionali}

Vengono elencati di seguito i requisiti funzionali, ossia quei requisiti legati alle funzionalità del sistema.


\subsection*{RF1 Registrazione}
Il sito dovrà presentare una pagina dedicata alla registrazione di un nuovo utente, ossia un utente anonimo, che visita il sito. Per effettuare la registrazione sarà necessaria una email valida e una password. Per completare la registrazione sarà inoltre necessaria la verifica tramite un'email di conferma.
Saranno inoltre richiesti altri dati personali quali nome e cognome.
\begin{enumerate} [leftmargin=40pt]
    \item Nel caso in cui i dati inseriti non siano validi, l'applicazione provvederà a mostrare un messaggio di errore per informare l'utente. 
\end{enumerate}


\subsection*{RF2 Login}
 Il sito presenterà una pagina dedicata all’accesso tramite credenziali (email-password) precedentemente registrate (RF1).
 Il sito web provvede quindi a verificare la correttezza dei dati inseriti per permettere ad un utente registrato di accedere all'area riservata.
 \begin{enumerate} [leftmargin=40pt]
    \item Qualora le credenziali siano errate, verrà mostrato un messaggio di errore per avvisare l'utente.
    \item La schermata di login dovrà presentare un link che permetta il reindirizzamento alla pagina "Recupera password", dove sarà possibile recuperare la propria password tramite una email
    \item Il sito, tramite l’apposita sezione nel proprio account (RF6.4), permetterà all’utente registrato di cambiare la propria password utilizzando l’email ricevuta per confermare l'operazione.
\end{enumerate}


\subsection*{RF3 Account Amministratore}
Nel sito web sarà presente un account amministratore, che avrà accesso a funzionalità privilegiate e poteri estesi all'interno del sito.\\
L'account admin sarà fondamentale per la gestione e il controllo completo del sito web. Di seguito sono elencate le principali caratteristiche e le responsabilità dell'account admin:
 \begin{enumerate} [leftmargin=40pt]
     \item \textbf{eliminare utenti registrati}:\newline Qualora un utente registrato non rispetti le norme del sito (RNF7) o voglia eliminare il proprio account, l'applicazione permetterà all'account amministrazione di eliminare l'account con i relativi dati personali inseriti al momento della registrazione.
     \item \textbf{eliminare una recensione}:\newline Il sito permetterà all'amministratore di eliminare una recensione nel caso in cui questa non rispetti le norme del sito (RNF7). L'operazione verrà notificata all'utente e nel caso in cui l'evento si ripeta eliminerà definitivamente l'account registrato. 
     \item \textbf{modificare e eliminare una montagna o un rifugio}:\newline
    Sarà permesso all'account amministratore di modificare le informazioni relative ad un rifugio o ad una montagna. Il sito permetterà inoltre di eliminare un rifugio o una montagna nel caso in cui le informazioni siano già presenti (ridondanza della pagina) oppure contenga contenuti inappropriati. 
    \item \textbf{rispondere alle richieste di supporto}:\newline
    In caso di richieste di supporto da parte di utenti anonimi o registrati, l'amministratore fornirà adeguato supporto per risolvere le difficoltà.
     \item \textbf{Inserire montagna}:\newline
    Il sito permetterà all'account amministratore di inserire una montagna. Dopo l'aggiunta di una montagna gli utenti registrati potranno aggiungere un rifugio legato ad essa. 
    
    
 
 \end{enumerate}
 L'account amministratore sarà già presente quando verrà rilasciata l'applicazione al cliente. 


\subsection*{RF4 Account registrato}
 Il sito presenterà una sezione per la visualizzazione dell'account registrato, dove sarà possibile:
 \begin{enumerate} [leftmargin=40pt]
     \item visionare i dati personali
     \item visionare le recensioni effettuate
     \item cambiare la password del proprio account
     \item chiedere l'eliminazione del proprio account
 \end{enumerate}


 \subsection*{RF5 Account anonimo}
 La pagina permetterà anche di gestire un utente anonimo, ossia un utente che non ha effettuato la registrazione al sito.\\
 Questo avrà funzionalità più limitate rispetto ad un utente registrato, infatti l'applicazione web gli permetterà di:
 \begin{enumerate} [leftmargin=40pt]
  \item visualizzare l'elenco dei rifugi e luoghi di spicco
  \item visualizzare l'elenco delle montagne
  \item visionare eventuali recensioni relative ad un rifugio o ad una montagna
\end{enumerate}

 
\subsection*{RF6 Profilo utente}
L'applicazione offrirà agli utenti registrati la possibilità di eseguire diverse operazioni relative alla gestione del proprio profilo personale. Le principali operazioni sono: 

\begin{enumerate} [leftmargin=40pt]
  \item \textbf{Caricare una foto profilo:} l'applicazione permetterà agli utenti di selezionare e caricare un'immagine che rappresenti il loro profilo. Questa foto sarà visibile agli altri utenti e contribuirà a personalizzare il loro profilo.

  \item \textbf{Modificare la foto profilo:} il sito permetterà, oltre al caricamento iniziale, di apportare modifiche alla foto del profilo quando lo si desidera. 

  

  \item \textbf{Visualizzare dati personali:} in questa sezione, gli utenti avranno accesso ai propri dati personali registrati nell'applicazione. Potranno visualizzare informazioni quali nome, cognome, indirizzo email e altre informazioni pertinenti.

  \item \textbf{Aggiornare dati personali:} l'applicazione permetterà inoltre di apportare modifiche e aggiornamenti ai seguenti dati, garantendo che le informazioni del proprio account siano sempre accurate e aggiornate
  
  \begin{enumerate} [leftmargin=40pt]
     \item Modificare nome e cognome
     \item Modificare password
     \item Modificare email
 \end{enumerate}

  \item \textbf{Elimina account:} nell’apposita sezione, l'applicazione consentirà a un utente registrato di effettuare la cancellazione dell'account grazie alla quale (previa conferma via email) sarà possibile eliminare le proprie informazioni personali.
\end{enumerate}


\subsection*{RF7 Recensioni}
L'applicazione deve permettere ad un utente registrato di poter recensire ciascun rifugio.\newline
In questo modo, il rating della recensione potrà poi essere usato nell'area di ricerca, per fornire risultati specifici all'utente che ha effettuato la ricerca. \newline\newline
\textbf{Gestione recensioni effettuate}: La web app darà la possibilità ad un utente che ha effettuato l'accesso al proprio account(RF2) di vedere tutte le recensioni da lui effettuate, elencate in ordine di data di pubblicazione.
Recandosi nella pagina in cui si è effettuata una recensione (RF7), l'applicazione permetterà inoltre all'utente registrato di modificare o eliminare una recensione precedentemente fatta.\newline


\subsection*{RF8 Ricerca}
L'applicazione fornirà agli utenti registrati e anonimi la possibilità di effettuare la ricerca di rifugi o luoghi di spicco grazie all’utilizzo di un’apposita barra di ricerca.\newline
Durante l'operazione di ricerca di un rifugio, l'applicazione consentirà all'utente di raffinare la sua ricerca tramite l'uso di filtri.\newline
Questi filtri rappresentano una serie di criteri che l'utente può selezionare per fare in modo che la ricerca produca dei risultati specifici.
Tali filtri potrebbero essere diversi parametri quali la posizione geografica, la distanza, le valutazioni degli utenti registrati, la difficoltà.


\subsection*{RF9 Visualizzazione montagne e rifugi}
Il sito web permetterà ad un utente registrato e non registrato di visualizzare l'elenco dei rifugi e luoghi di spicco.\newline
Inoltre, tramite un'apposita pagina, l'applicazione web permetterà di visionare i dettagli di un rifugio.\newline
Il sistema fornirà un'ulteriore pagina per la visualizzazione delle montagne.


\subsection*{RF10 Aggiungere rifugio}
Il sito permetterà agli utenti registrati di contribuire attivamente all'espansione del database, consentendo loro di aggiungere nuovi rifugi o luoghi di notevole interesse. Questa funzionalità svolge un ruolo fondamentale nel mantenere il sito costantemente arricchito e aggiornato, garantendo che sia una risorsa informativa completa e sempre in evoluzione.

\subsection*{RF11 Supporto}
Il sito web deve includere una pagina dedicata al supporto, fornendo agli utenti registrati un mezzo efficace per richiedere assistenza o informazioni aggiuntive. Le richieste di supporto effettuate tramite questa pagina verranno automaticamente inoltrate all'amministratore (RF4).

\subsection*{RF12 Cambio lingua}
Il sito web deve offrire un menu o un'opzione chiaramente visibile nella parte superiore, facilmente accessibile, che consenta agli utenti di selezionare la lingua di loro preferenza.\\
Questo menu deve elencare tutte le lingue supportate, quali italiano, inglese, tedesco.


\subsection*{RF13 Meteo sulla Montagna}
Il sito deve fornire la possibilità di accedere alle informazioni meteo relative a una specifica montagna direttamente dalla pagina dedicata a quest'ultima. Questa funzionalità consentirà agli utenti di pianificare in modo efficace le loro attività in montagna, tenendo conto delle condizioni meteorologiche previste





\subsection*{Tabella riassuntiva}

Di seguito è presente una tabella riassuntiva che associa ad ogni obiettivo, precedentmente descritto, i requisiti funzionali che permettono di raggiungerlo.

\begin{center}
\begin{tabular}{|l|l|}
\hline
Obiettivi     &Requisiti funzionale          \\
\hline
O1            & RF9, RF5\\ \hline 
O2            & RF1, RF2, RF4, RF7\\ \hline 
O3            & RF8, RF5\\ \hline 
O4            & RF3, RF11\\\hline
O5            & RF10\\\hline 
\end{tabular}
\newpage
\end{center}

\section{Requisiti non funzionali}

Vengono di seguito elencati e descritti i requisiti non funzionali del progetto. Ciascun requisito diversamente dai requisiti funzionali è legato alle performance.

\subsection*{RNF1 Sicurezza}
\begin{enumerate}[label = \alph*]
    \item Le sezioni dedicate agli utenti registrati (RF4) devono essere protette da accessi non autorizzati, tramite l’utilizzo di indirizzo email e password. 
    La password deve rispettare determinati criteri di sicurezza, quali: 
        \begin{itemize}
            \item Una lunghezza minima di 8 caratteri
            \item Lettere sia maiuscole che minuscole
            \item Presenza di almeno un carattere speciale (\&, £, \$, \#, ...)
            \item Presenza di almeno un numero 
        \end{itemize}
    \item Tutti i dati sensibili degli utenti, come informazioni personali e password, saranno crittografati sia in transito che a riposo, ossia quando il dato non viene utilizzato per l'accesso. Saranno implementati protocolli di crittografia standard del settore come TLS/SSL.
    \item Per aumentare la sicurezza, utilizzeremo un servizio di invio di email di verifica. Quest'ultimo consente di inviare un'email di conferma prima di procedere con la creazione o l'eliminazione di un profilo utente e prima di autorizzare modifiche relative all'indirizzo email e alla password di un account. 
\end{enumerate}


\subsection*{RNF2 Performance}
Il sito deve essere in grado di gestire il numero richiesto di utenti senza alcun degrado delle prestazioni. Deve essere inoltre veloce e rendere gradibile la navigazione ad un utente mobile, di conseguenza il tempo di risposta dopo un'operazione dell'utente sarà di massimo 2 secondi. \footnote{"Se il caricamento di un sito Web richiede molto tempo, ciò può avere effetti negativi sull'esperienza dell'utente, sul traffico del sito e sulla SEO (Search Engine Optimization). I siti Web ottimizzati per le prestazioni hanno un vantaggio rispetto ai siti Web lenti."    Cloudflare }


\subsection*{RNF3 Compatibilità}
Il sito web dovrà essere compatibile con le ultime versioni dei seguenti browser e dispositivi:
\begin{itemize}
    \item Browser desktop: Google Chrome (v. 106 o successive), Mozilla Firefox(v. 104 o successive), Microsoft Edge(v. 108 o successive), Safari (v. 13 o successive).
     \item Browser mobile: Safari su iOS(v. 15 o successive), Google Chrome su Android(v. 110 o successive).
\end{itemize}


\subsection*{RNF4 Affidabilità e disponibilità}
Per poter rendere affidabile la web app, i dati inseriti dagli utenti verranno salvati all’interno di un database. Su questi verranno effettuati backup, settimanali, per poter eventualmente recuperare i dati in caso di guasti del sistema o di attacchi informatici.
\\La disponibilità del sito è un elemento fondamentale che garantisce agli utenti l'accesso senza interruzioni in qualsiasi momento.


\subsection*{RNF5 Usabilità}
Il sito web deve essere progettato in modo tale da risultare intuitivo e accessibile, garantendo una facile fruibilità per un pubblico ampio, composto principalmente da individui con un'età compresa tra i 15 e i 65 anni.\\
Il sito è stato progettato in modo che dopo 20 minuti l'utente sia in grado di saper utilizzare tutte le funzionalità senza leggere le istruzioni.\\
Questo pubblico variegato potrà accedere e utilizzare il sito sia tramite dispositivi mobili come smartphone e tablet, sia attraverso computer desktop e portatili, al fine di massimizzare la sua adozione e utilizzo in modo versatile e inclusivo.


\subsection*{RNF6 Norme recensioni}
Per garantire un ambiente informativo, costruttivo e rispettoso, abbiamo stabilito le seguenti linee guida per le recensioni (RF7) degli utenti registrati:
\begin{itemize}
    \item Le recensioni dovrebbero essere pertinenti a quanto contenuto all'interno della pagina visualizzata
    \item Non inserire commenti offensivi, linguaggio volgare, discriminazione, insulti o minacce verso individui o gruppi.\\
    In caso l'utente admin trovi recensioni non adeguate, provvederà a cancellarla.
    \item La pubblicazione ripetuta di contenuti simili o allo scopo di promuovere un'agenda personale non sarà consentito.
\end{itemize}

\subsection*{RNF7 Gestione delle immagini}
Le immagini devono essere ottimizzate per ridurre i tempi di caricamento e migliorare le prestazioni. Se non gestite correttamente, le immagini possono rallentare significativamente il caricamento del sito e avere un impatto negativo sull'esperienza dell'utente.
La dimensione massima consentita per un'immagine sarà 3MB


\subsection*{RNF8 Privacy}
Il sito web dovrà essere conforme alle seguenti normative e regolamenti:
\begin{itemize}
    \item GDPR (General Data Protection Regulation): il sito raccoglierà e tratterà dati personali degli utenti, nome, cognome, foto dell'utente e email. Saranno forniti avvisi chiari sulla privacy e acquisiti consensi espliciti per la raccolta e l'uso dei dati.
    \item Leggi sul copyright: sarà rispettato il copyright per tutti i contenuti utilizzati sul sito, evitando l'uso non autorizzato di testi, immagini e altri materiali protetti.
    \item COPPA (Children's Online Privacy Protection Act): il sito effettuerà un controllo dell'età dell'utente che si registra, impedendo la registrazione ai minori di 13 anni.
\end{itemize}


\subsection*{RNF9 Aggiornamenti} 
Il sito web sarà progettato e sviluppato con una solida architettura e un'infrastruttura scalabile al fine di garantire la sua capacità di supportare senza problemi eventuali futuri aggiornamenti.\\
Ogni mese verrà controllato se l'architettura è adeguata, e in caso verranno effettuati opportuni aggiornamenti.\\
Questi aggiornamenti potrebbero essere di natura diversa, tra cui la risoluzione di problemi, l'aggiunta di nuove funzionalità e il miglioramento complessivo dell'esperienza dell'utente.
 


\newpage
\section{Front-end}
Verranno mostrate di seguito le interfacce delle pagine web con relative descrizioni.
\subsection*{FE1 Homepage}
\begin{figure}[ht]
   \centering
    \includegraphics[width=0.6\textwidth]{img/Homepage.png}
    \caption{Homepage}
\end{figure}
Questa sarà la web page principale, da cui sarà possibile recarsi nelle seguenti pagine:
\begin{itemize}
    \item Profilo (RF6, nella barra di navigazione in alto a destra)
    \item Pagina delle montagne (RF9) (sotto il nome del sito)
    \item Pagina rifugi (RF9) (sotto il nome del sito)
\end{itemize}


\subsection*{FE2 Sign-up page}
\begin{figure}[H]
  \begin{subfigure}{0.49\textwidth}
    \centering
    \includegraphics[width=\textwidth]{img/Sign-up 1.png} 
    \caption{Sign-up 1 pagina}
  \end{subfigure}
  \hfill % Spazio vuoto orizzontale tra le due immagini
  \begin{subfigure}{0.49\textwidth}
    \centering
    \includegraphics[width=\textwidth]{img/Sign-up 2.png} 
    \caption{Sign-up 2 pagina}
  \end{subfigure}
  \caption{Pagine di Sign-up (RF1)}
\end{figure}
All'interno di queste sezioni dedicate, sarà possibile compiere il processo di registrazione (RF1) per l'ingresso di nuovi utenti nel sito web. Nel primo passo di questo processo, verrà chiesto di fornire alcune informazioni fondamentali per la creazione dell'account personale, oppure sarà possibile registrarsi tramite il proprio account Google, da cui il sito estrapolerà tutte le informazioni necessarie per la registrazione.

Nella prima sezione del modulo di registrazione, ci sarà la possibilità di inserire il nome e cognome. Questo passo iniziale è essenziale per garantire che l'account sia personalizzato e riconoscibile. Successivamente, si potrà procedere al passo successivo del processo, dove si avrà l'opzione di fornire ulteriori dettagli relativi al proprio account.

Nel secondo form, avrai la possibilità di inserire informazioni quali l'indirizzo email, che si intende associare al proprio account, e la password che si desidera utilizzare per l'accesso. L'email svolgerà un ruolo cruciale come mezzo di comunicazione e recupero dell'account, mentre la password garantirà la sicurezza dei tuoi dati personali (RNF1).





\subsection*{FE3 Log-in page}
\begin{figure}[H]
   \centering
    \includegraphics[width=0.6\textwidth]{img/Log-in.png}
    \caption{Pagina di log-in (RF2)}
\end{figure}
Questa pagina permetterà all'utente anonimo di effettuare il login (RF2) tramite la propria email e la password, oppure sarà possibile accedere tramite il proprio account Google. Una volta inseriti basterà cliccare "Accedi" per effettuare il login. Nel caso in cui l'untente non dovesse ricordarsi la password potrà, cliccando il testo: "Password dimenticata?", richiedere una nuova password che gli sarà inviata via email all'inidirizzo fornito durante la registrazione.


\subsection*{FE4 Pagina montagne}
\begin{figure}[H]
   \centering
    \includegraphics[width=0.6\textwidth]{img/Pagina montagne.png}
    \caption{Pagina elenco montagne (RF9)}
\end{figure}
Questa pagina darà la possibilità a qualunque tipo di utente di visionare un elenco con tutte le montagne della zona. Permetterà la ricerca di un rifugio o di una montagna tramite la barra di ricerca in alto e sarà possibile applicare dei filtri per avere risultati pertinenti a quanto si deve cercare. La freccia in basso a destrà perfetterà di tornare nella homepage.


\subsection*{FE5 Pagina account}
\begin{figure}[H]
   \centering
    \includegraphics[width=0.6\textwidth]{img/Pagina account.png}
    \caption{Pagina account personale (RF6)}
\end{figure}


Questa pagina permetterà all'utente registrato di:
\begin{itemize}
    \item cambiare le informazioni personali (RF6);
    \item cambiare la password (RF6.4);
    \item richiedere l'eliminazione dell'account (RF6.5);
    \item visualizzare le recensioni effettuate con il proprio account(RF9);
\end{itemize}

\subsection*{FE5 Pagina rifugio (RF9)}
\begin{figure}[H]
   \centering
    \includegraphics[width=0.6\textwidth]{img/Pagina rifugio.png}
    \caption{Pagina rifugio}
\end{figure}
Questa pagina permetterà di visualizzare un rifugio e le sue informazioni (RF9). Inoltre, scorrendo la pagina verso il basso si troverà una sezione in cui effettuare le recensioni (RF7) e visualizzare le recensioni effettuate da altri utenti registrati.


\newpage
\section{Back-end}

Descriviamo il sito dal punto di vista dei servizi con i quale interagisce.

\subsection*{BE1 Gestore Email}
Il sito si appoggierà ad un sistema che permetterà di inviare un'email e verificare un utente che si deve registrare. Nell'email sarà presente un link per attivazione dell'account. L'utente dovrà quindi cliccare il link e loggarsi per iniziare ad utilizzare il sito.


\subsection*{BE2 Google Login API}
Il sito permetterà di registrarsi e loggarsi tramite un servizio esterno. Al momeno della registrazioni (RF1) o del login(RF2), sarà presente un pulsante per entrare utilizzando il proprio account google. Cliccato il pulsante l'utente verrà reindirizzato alla pagina di google dovre potrà inserire le proprie credenziali.

\subsection*{BE3 Google Maps API}
La web app si appoggierà alle API di Google Maps per mostrare i rifugi inseriti degli utenti(RF10). Sarà presente una mappa interattiva che mostrerà l'area circostante al rifugio.


\subsection*{BE4 Google Weather API}
Il sito sfrutterà le API di Google Weather per permettere la visualizzazione del meteo di una montagna (RF13). Questa integrazione consentirà agli utenti di accedere in modo agevole e affidabile alle informazioni meteorologiche dettagliate relative a ciascuna montagna, garantendo una previsione precisa delle condizioni meteorologiche future.
\begin{figure}[H]
   \centering
    \includegraphics[width=0.5\textwidth]{img/Backend-diagram.png}
    \caption{Diagramma back-end}
\end{figure}


\end{document}